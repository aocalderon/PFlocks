\documentclass[12pt]{scrartcl}
\usepackage[square]{natbib}

\title{Parallel detection of movement flock patterns in large trajectory databases}
\author{_}

\begin{document}

\maketitle
 
\section{Introduction}
Technology advances in last past decades have triggered an explosion in the capture of spatio-temporal data.  The increase popularity of GPS devices and smart-phones together with the emerging of new disciplines such as the Internet of Things and Satellite/UAS high-resolution imagery have made possible to collect vast amounts of data with a spatial and temporal component attached to them.

Together with this, the interest to extract valuable information from such large databases has also appeared.  Spatio-temporal queries about most popular places or frequent events are still useful, but more complex patterns are recently shown an increase of interest.  In particular, those what describe group behaviour of moving objects through significant periods of time.  Moving cluster \citep{kalnis_discovering_2005}, convoys \citep{jeung_discovery_2008}, flocks \citep{gudmundsson_computing_2006} and swarm patterns \citep{li_swarm:_2010} are new movement patterns which unveil how entities move together during a minimum time interval. 

Applications for this kind of information are diverse and interesting, in particular if they come in the way of trajectory datasets \citep{jeung_trajectory_2011, huang_mining_2015}. Case of studies range from transportation system management and urban planning \citep{di_lorenzo_allaboard:_2016} to Ecology \citep{la_sorte_convergence_2016}.  For instance, \cite{turdukulov_visual_2014} explore the finding of complex motion patterns to discover similarities between tropical cyclone paths.  Similarly, \cite{amor_persistence_2016} use eye trajectories to understand which strategies people use during a visual search. Also, \cite{holland_movements_1999} track the behavior of tiger sharks in the coasts of Hawaii in order to understand their migration patters.

In particular, a moving flock pattern show how objects move close enough during a given time period.  Closeness is defined by a disk of a given radius where the entities must keep inside.  Given that the disk can be situated at any location, it is not a trivial problem.  In deed, \cite{gudmundsson_computing_2006} claims that find flock patterns where the same entities stay together during their duration is a NP-hard problem. \cite{vieira_-line_2009} proposed the BFE algorithm which the first approach to be able to detect flock patterns in polynomial time.

Despite the fact that much more data become available, state-of-the-art techniques to mine complex movement patterns still depict poor performance for big spatial data problems.  This work presents a parallel algorithm to discover moving flock patterns in large trajectory databases.  It is thought that new trends in distributed in-memory frameworks for spatial operations could help to speed up the detection of this kind of patterns.

\bibliographystyle{plainnat}
\bibliography{pflock}

\end{document}