%%
%% This is file `sample-manuscript.tex',
%% generated with the docstrip utility.
%%
%% The original source files were:
%%
%% samples.dtx  (with options: `manuscript')
%% 
%% IMPORTANT NOTICE:
%% 
%% For the copyright see the source file.
%% 
%% Any modified versions of this file must be renamed
%% with new filenames distinct from sample-manuscript.tex.
%% 
%% For distribution of the original source see the terms
%% for copying and modification in the file samples.dtx.
%% 
%% This generated file may be distributed as long as the
%% original source files, as listed above, are part of the
%% same distribution. (The sources need not necessarily be
%% in the same archive or directory.)
%%
%% Commands for TeXCount
%TC:macro \cite [option:text,text]
%TC:macro \citep [option:text,text]
%TC:macro \citet [option:text,text]
%TC:envir table 0 1
%TC:envir table* 0 1
%TC:envir tabular [ignore] word
%TC:envir displaymath 0 word
%TC:envir math 0 word
%TC:envir comment 0 0
%%
%%
%% The first command in your LaTeX source must be the \documentclass command.
%%%% Small single column format, used for CIE, CSUR, DTRAP, JACM, JDIQ, JEA, JERIC, JETC, PACMCGIT, TAAS, TACCESS, TACO, TALG, TALLIP (formerly TALIP), TCPS, TDSCI, TEAC, TECS, TELO, THRI, TIIS, TIOT, TISSEC, TIST, TKDD, TMIS, TOCE, TOCHI, TOCL, TOCS, TOCT, TODAES, TODS, TOIS, TOIT, TOMACS, TOMM (formerly TOMCCAP), TOMPECS, TOMS, TOPC, TOPLAS, TOPS, TOS, TOSEM, TOSN, TQC, TRETS, TSAS, TSC, TSLP, TWEB.
% \documentclass[acmsmall]{acmart}

%%%% Large single column format, used for IMWUT, JOCCH, PACMPL, POMACS, TAP, PACMHCI
% \documentclass[acmlarge,screen]{acmart}

%%%% Large double column format, used for TOG
% \documentclass[acmtog, authorversion]{acmart}

%%%% Generic manuscript mode, required for submission
%%%% and peer review
\documentclass[sigconf,screen,review]{acmart}
%% Fonts used in the template cannot be substituted; margin 
%% adjustments are not allowed.
%%
%% \BibTeX command to typeset BibTeX logo in the docs
\AtBeginDocument{%
  \providecommand\BibTeX{{%
    \normalfont B\kern-0.5em{\scshape i\kern-0.25em b}\kern-0.8em\TeX}}}

%% Rights management information.  This information is sent to you
%% when you complete the rights form.  These commands have SAMPLE
%% values in them; it is your responsibility as an author to replace
%% the commands and values with those provided to you when you
%% complete the rights form.
\setcopyright{acmcopyright}
\copyrightyear{2023}
\acmYear{2023}
\acmDOI{XXXXXXX.XXXXXXX}

%% These commands are for a PROCEEDINGS abstract or paper.
\acmConference[Conf '23]{
  %International Symposium on Spatial and Temporal Data
}{
  %August 23--25, 2023
}{
  %Calgary, Canada
}
%
%  Uncomment \acmBooktitle if th title of the proceedings is different
%  from ``Proceedings of ...''!
%
\acmBooktitle{
%SSTD '23: International Symposium on Spatial and Temporal Data, August 23--25, 2023, Calgary, Canada
}
\acmPrice{15.00}
\acmISBN{978-1-4503-XXXX-X/18/06}


%%
%% Submission ID.
%% Use this when submitting an article to a sponsored event. You'll
%% receive a unique submission ID from the organizers
%% of the event, and this ID should be used as the parameter to this command.
%%\acmSubmissionID{123-A56-BU3}

%%
%% end of the preamble, start of the body of the document source.

%\usepackage{titlesec}
%\titlespacing*{\section}{5pt}{0.1\baselineskip}{0.2\baselineskip}

\usepackage{subcaption}
%% For algorithms...
\usepackage{algorithm}
\usepackage{algpseudocode}
\algnewcommand\algorithmicforeach{\textbf{for each}}
\algdef{S}[FOR]{ForEach}[1]{\algorithmicforeach\ #1\ \algorithmicdo}
\algnewcommand\algorithmicswitch{\textbf{switch}}
\algnewcommand\algorithmiccase{\textbf{case}}
\algdef{SE}[SWITCH]{Switch}{EndSwitch}[1]{\algorithmicswitch\ #1\ \algorithmicdo}{\algorithmicend\ \algorithmicswitch}
\algdef{SE}[CASE]{Case}{EndCase}[1]{\algorithmiccase\ #1}{\algorithmicend\ \algorithmiccase}
\algtext*{EndSwitch}
\algtext*{EndCase}

\setlength{\textfloatsep}{2pt plus 1.0pt minus 2.0pt}

\usepackage{tikz}
\usetikzlibrary{shapes, arrows}

\begin{document}

%%
%% The "title" command has an optional parameter,
%% allowing the author to define a "short title" to be used in page headers.
\title{Parallel finding of moving flock patterns}

%%
%% The "author" command and its associated commands are used to define
%% the authors and their affiliations.
%% Of note is the shared affiliation of the first two authors, and the
%% "authornote" and "authornotemark" commands
%% used to denote shared contribution to the research.

\author{Andres Calderon-Romero}
%\authornote{All authors contributed equally to this research.}
\email{acald013@ucr.edu}
\affiliation{
  \institution{University of California, Riverside}
  \country{}
  \city{}
}

\author{Petko Balakov}
\email{petko@}
\affiliation{
  \institution{Esri}
  \country{}
  \city{}
}

\author{Marcos Vieira}
\email{marcos@}
\affiliation{
  \institution{Google}
  \country{}
  \city{}
}

\author{Vassilis J. Tsotras}
\email{tsotras@cs.ucr.edu}
\affiliation{
  \institution{University of California, Riverside}
  \country{}
  \city{}
}

%%
%% By default, the full list of authors will be used in the page
%% headers. Often, this list is too long, and will overlap
%% other information printed in the page headers. This command allows
%% the author to define a more concise list
%% of authors' names for this purpose.
\renewcommand{\shortauthors}{Calderon, et al.}

%%
%% The abstract is a short summary of the work to be presented in the
%% article.
\begin{abstract}

\end{abstract}

%%
%% The code below is generated by the tool at http://dl.acm.org/ccs.cfm.
%% Please copy and paste the code instead of the example below.
%%
\begin{CCSXML}
<ccs2012>
   <concept>
       <concept_id>10010147.10010169.10010170</concept_id>
       <concept_desc>Computing methodologies~Parallel algorithms</concept_desc>
       <concept_significance>500</concept_significance>
       </concept>
   <concept>
       <concept_id>10002951.10002952.10002971</concept_id>
       <concept_desc>Information systems~Data structures</concept_desc>
       <concept_significance>500</concept_significance>
       </concept>
   <concept>
       <concept_id>10010147.10010919.10010172.10003817</concept_id>
       <concept_desc>Computing methodologies~MapReduce algorithms</concept_desc>
       <concept_significance>500</concept_significance>
       </concept>
 </ccs2012>
\end{CCSXML}

\ccsdesc[500]{Computing methodologies~Parallel algorithms}
\ccsdesc[500]{Information systems~Data structures}
\ccsdesc[500]{Computing methodologies~MapReduce algorithms}

%%
%% Keywords. The author(s) should pick words that accurately describe
%% the work being presented. Separate the keywords with commas.
\keywords{Mobile patterns}

%%\received{20 February 2007}
%%\received[revised]{12 March 2009}
%%\received[accepted]{5 June 2009}

%%
%% This command processes the author and affiliation and title
%% information and builds the first part of the formatted document.
\maketitle

\section{Introduction}
Technology advances in last past decades have triggered an explosion in the capture of spatio-temporal data.  The increase popularity of GPS devices and smart-phones together with the emerging of new disciplines such as the Internet of Things and Satellite/UAS high-resolution imagery have made possible to collect vast amounts of data with a spatial and temporal component attached to them.

Together with this, the interest to extract valuable information from such large databases has also appeared.  Spatio-temporal queries about most popular places or frequent events are still useful, but more complex patterns are recently shown an increase of interest.  In particular, those what describe group behaviour of moving objects through significant periods of time.  Moving cluster \cite{kalnis_discovering_2005}, convoys \cite{jeung_discovery_2008}, flocks \cite{gudmundsson_computing_2006} and swarm patterns \cite{li_swarm:_2010} are new movement patterns which unveil how entities move together during a minimum time interval. 

Applications for this kind of information are diverse and interesting, in particular if they come in the way of trajectory datasets \cite{jeung_trajectory_2011, huang_mining_2015}. Case of studies range from transportation system management and urban planning \cite{di_lorenzo_allaboard:_2016} to Ecology \cite{la_sorte_convergence_2016}.  For instance, \cite{turdukulov_visual_2014} explore the finding of complex motion patterns to discover similarities between tropical cyclone paths.  Similarly, \cite{amor_persistence_2016} use eye trajectories to understand which strategies people use during a visual search. Also, \cite{holland_movements_1999} tracks the behavior of tiger sharks in the coasts of Hawaii in order to understand their migration patters.

In particular, a moving flock pattern show how objects move close enough during a given time period.  Closeness is defined by a disk of a given radius where the entities must keep inside.  Given that the disk can be situated at any location, it is not a trivial problem.  In deed, \cite{gudmundsson_computing_2006} claims that find flock patterns where the same entities stay together during their duration is a NP-hard problem. \cite{vieira_2009} proposed the BFE algorithm which the first approach to be able to detect flock patterns in polynomial time.

Despite the fact that much more data become available, state-of-the-art techniques to mine complex movement patterns still depict poor performance for big spatial data problems.  This work presents a parallel algorithm to discover moving flock patterns in large trajectory databases.  It is thought that new trends in distributed in-memory frameworks for spatial operations could help to speed up the detection of this kind of patterns.

\section{Related work}
Recently increase use of location-aware devices (such as GPS, Smart phones and RFID tags) has allowed the collection of a vast amount of data with a spatial and temporal component linked to them.  Different studies have focused in analyzing and mining this kind of collections \cite{leung_knowledge_2010, miller_geographic_2001}.  In this area, trajectory datasets have emerged as an interesting field where diverse kind of patterns can be identified \cite{zheng_computing_2011, vieira_spatio-temporal_2013}.  For instance, authors have proposed techniques to discover motion spatial patterns such as moving clusters \cite{kalnis_discovering_2005}, convoys \cite{jeung_discovery_2008} and flocks \cite{benkert_reporting_2008, gudmundsson_computing_2006}.  In particular, \cite{vieira_2009} proposed BFE (Basic Flock Evaluation), a novel algorithm to find moving flock patterns in polynomial time over large spatio-temporal datasets. 

A flock pattern is defined as a group of entities which move together for a defined lapse of time \cite{benkert_reporting_2008}.  Applications to this kind of patterns are rich and diverse.  For example, \cite{calderon_romero_mining_2011} finds moving flock patterns in iceberg trajectories to understand their movement behavior and how they related to changes in ocean's currents. 

The BFE algorithm presents an initial strategy in order to detect flock patterns.  In that, first it finds disks with a predefined diameter ($\varepsilon$) where moving entities could be close enough at a given time interval.  This is a costly operation due to the large number of points and intervals to be analyzed ($\mathcal{O}(2n^2)$ per time interval).  The technique uses a grid-based index and a stencil to speed up the process, but the complexity is still high.

\cite{calderon_romero_mining_2011} and \cite{turdukulov_visual_2014} use a frequent pattern mining approach to improve performance during the combination of disks between time intervals.  Similarly, \cite{tanaka_improved_2016} introduce the use of plane sweeping along with binary signatures and inverted indexes to speedup the same process.  However, the above-mentioned methods still keep the same strategy as BFE to find the disks at each interval.  

\cite{arimura_finding_2014} and \cite{geng_enumeration_2014} use depth-first algorithms to analyze the time intervals of each trajectory to report maximal duration flocks.  However, these techniques are not suitable to find patterns in an on-line fashion.

Given the high complexity of the task, it should not be surprising the use of parallelism to increase performance.  \cite{fort_parallel_2014} use extreme and intersection sets to report maximal, longest and largest flocks on the GPU with the limitations of its memory model.  

Indeed, despite the popularity of cluster computing frameworks (in particular whose supporting spatial capabilities \cite{eldawy_spatialhadoop:_2014, yu_demonstration_2016, pellechia_geomesa:_2015-1, xie_simba:_2016-1}) there are not significant advances in this area.  At the best of our knowledge, this work is the first to explore in-memory distributed systems towards the detection of moving flock patterns.


\input{methods2}

%%
%% The acknowledgments section is defined using the "acks" environment
%% (and NOT an unnumbered section). This ensures the proper
%% identification of the section in the article metadata, and the
%% consistent spelling of the heading.
%\begin{acks}
%\end{acks}

%%
%% The next two lines define the bibliography style to be used, and
%% the bibliography file.
\balance
\bibliographystyle{ACM-Reference-Format}
\bibliography{pflocks}
\newpage 
\appendix

\section{Center computation.}
\begin{algorithm}
    \caption{Find the centers of given radius which circumference laid on the two input points.}
    \begin{algorithmic}[1]
        \Require Radius $\frac{\varepsilon}{2}$ and points $p_1$ and $p_2$.
        \Ensure Centers $c_1$ and $c_2$.
        
        \Function{FindCenters}{$p_1$, $p_2$, $\frac{\varepsilon}{2}$}
        \State $r^2 \gets (\frac{\varepsilon}{2})^2$
        \State $X \gets p_1.x - p_2.x$
        \State $Y \gets p_1.y - p_2.y$
        \State $d^2 \gets X^2 + Y^2$
        \State $R \gets \sqrt{\lvert 4 \times \frac{r^2}{d^2} - 1 \rvert}$
        \State $c_1.x \gets X + \frac{Y \times R}{2} + p_2.x$
        \State $c_1.y \gets Y - \frac{X \times R}{2} + p_2.y$
        \State $c_2.x \gets X - \frac{Y \times R}{2} + p_2.x$
        \State $c_2.y \gets Y + \frac{X \times R}{2} + p_2.y$
        
        \State \Return $c_1$ and $c_2$
        \EndFunction
    \end{algorithmic}
    \label{app:centers}
\end{algorithm}

\section{Disk pruning.}
\begin{algorithm}
    \caption{Prune disks which are duplicate or subset of others.}
    \begin{algorithmic}[1]
        \Require Set of disks $D$.
        \Ensure Set of disks $D^{\prime}$ without duplicate or subsets.
        
        \Function{PruneDisks}{$D$}
        \State $E \gets \varnothing$
        \ForAll{disk $d_i$ in $D$}
            \State $N \gets d_i \cap D$
            \ForAll{disk $n_j$ in $N$}
                \If{$d_i$ contains all the elements of $n_j$}
                        \State $E \gets E \cup {n_j}$
                \EndIf
            \EndFor
        \EndFor        
        \State $D^{\prime} \gets D \setminus E$
        \State \Return $D^{\prime}$
        \EndFunction
    \end{algorithmic}
    \label{app:disks}
\end{algorithm}


\end{document}
